This is the project for the E\+P\+FL class Programming Concepts in Scientific Computing -\/ Master 1 Class -\/ Computer Science and Engineering

The program could be use simply by importing the project repository in C\+Lion and to run the project. Once the project. was launched one has the possibility to chose between execut-\/ ing the program by using a manual input in the terminal (by hitting 1) and using the information saved in the config.\+dat file (by hitting 2). However both ways of doing lead to the same functionalities. The user could chose between five different options by simply entering a number between 1 and 5 in the C\+Lion terminal. As explicitly mentioned in the C\+Lion terminal the user could use 1. to perform a Least Squares data approximation, type 2. to appreciate the graphs of Fourier data approximation, 3. to compute the \mbox{\hyperlink{class_lagrange}{Lagrange}} polynomial data approximation, 4. for the Piece\+Wise Least Square data approximation and finally 5. for the Piece\+Wise \mbox{\hyperlink{class_lagrange}{Lagrange}} data approximation. The config.\+dat file contained four lines. The first field, labeled with Approximation Method, should contain a number between 1 and 5. In this line the user could choose between the five possible approximation techniques. More, the second field was used to specific with function the user wanted to interpolate. The third and fourth fields were not necessarily used by all the approximation methods. For instance the Fourier approximation did not need these last two lines so one could give random numbers in these containers (as long as there were some values). This was also true for the \mbox{\hyperlink{class_lagrange}{Lagrange}} approximation code. The other methods, like least Square, \mbox{\hyperlink{class_lagrange}{Lagrange}} by single pieces needed the first three lines in order to specify the degree of the polynomial function to interpolate. Finally, Least Square approximation by pieces needed all the four lines. In this specific case, all four fields were necessary since the method needed to know the degree and the number of intervals to compute. By doing all the steps as mentioned in the two previous paragraphs all the programs should work well and give a qualified approximation of the function. During the execution of the program he user had to follow the instructions on the terminal. However, it is important to know the pipeline of execution, meaning that once everything was entered the program took a few second to compute the answer. Following this step a graph appeared on the screen for 20 seconds. Once the first graph disappeared the testing part of the program was launched and showed a graph again but this time only comparing the expected data with the interpolation. This lasted for 20 seconds again and was then followed by a displaying of the accuracy of the prediction in comparison to the real data. 