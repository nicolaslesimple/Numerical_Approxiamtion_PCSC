This project helps user to approximate data that he can her-\//himself plug into the program using the data.\+dat file.

Currently two types of shapes can be drawn\+: The most important application is in data fitting. The best fit in the least-\/squares sense minimizes the sum of squared residuals (a residual being\+: the difference between an observed value, and the fitted value provided by a model). When the problem has substantial uncertainties in the independent variable (the x variable), then simple regression and least-\/squares methods have problems; in such cases, the methodology required for fitting errors-\/in-\/variables models may be considered instead of that for least squares. The sum of the squares of the offsets is used instead of the offset absolute values because this allows the residuals to be treated as a continuous differentiable quantity. However, because squares of the offsets are used, outlying points can have a disproportionate effect on the fit, a property which may or may not be desirable depending on the problem at hand. Least-\/squares problems fall into two categories\+: linear or ordinary least squares and nonlinear least squares, depending on whether or not the residuals are linear in all unknowns. The most common application of the least squares method, referred to as linear or ordinary, aims to create a straight line that minimizes the sum of the squares of the errors generated by the results of the associated equations, such as the squared residuals resulting from differences in the observed value and the value anticipated based on the model. The coefficients and summary outputs explain the dependence of the variables being tested. Currently three major types of approximations can be performed\+:


\begin{DoxyItemize}
\item \mbox{\hyperlink{LagrangePolynomial}{How to use the Lagrange method?}} Functions \char`\"{}\+How to use the Lagrange method?\char`\"{}
\item \mbox{\hyperlink{FourierTransforms}{How to does the Fourier Approximation works?}}
\item \mbox{\hyperlink{LeastSquares}{How to does the Least\+Sqaures Approximation works?}} 
\end{DoxyItemize}